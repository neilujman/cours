\documentclass[10pt,a4paper]{article}
\usepackage[utf8]{inputenc}
%\usepackage[french]{babel}
\usepackage[T1]{fontenc}
\usepackage{amsmath}
\usepackage{amsfonts}
\usepackage{amssymb}
\usepackage[left=2cm,right=2cm,top=2cm,bottom=2cm]{geometry}
\author{Julien RITON}
\title{Sur les équations différentielles}
\date{12/09/2022}
\newtheorem{dfn}{Définition}
\newtheorem{thm}{Théorème}
\usepackage{graphicx}
\begin{document}
\maketitle
\section{Retour sur la dérivation}
\subsection{Fonction numérique de la variable réelle}
Soient $I$ un intervalle ouvert, $f:I\rightarrow \mathbb{R}$ une fonction et $x_0$ un élément de $I$.
\begin{dfn}
On dit que $f$ est dérivable en $x_0$ si $\dfrac{f(x)-f(x_0)}{x-x_0}$ a une limite finie quand $x$ tend vers $x_0$.

La limite $\lim_{x\rightarrow x_0} \dfrac{f(x)-f(x_0)}{x-x_0}$ est notée $f'(x_0)$ et s'appelle le nombre dérivée de $f$ en $x_0$.

On dit que $f$ est dérivable sur $I$ si, quel que soit $x_0\in I$, $f$ est dérivable en $x_0$.

Dans ce, la fonction $f':I\rightarrow \mathbb{R}$ qui à $x$ associe $f'(x)$, s'appelle la dérivée de $f$.
\end{dfn}


\paragraph{L'usage de la dérivation}
\subparagraph{La dérivation}
Nous rappelons la définition de la dérivée d'une fonction $f$ en un point $x$.
C'est la limite quand $h$ tend vers $0$ du rapport des accroissements entre la valeur de la fonction et la variable et est noté $f'(x)$:
$$
f'(x)=\lim_{h\rightarrow 0} \dfrac{f(x+h)-f(x)}{(x+h)-x}
=\lim_{h\rightarrow 0} \dfrac{f(x+h)-f(x)}{h}
$$
On a aussi les notations suivantes:
$$f'x), \; \dfrac{df(x)}{dx} $$


\subsection{Fonctions vectorielles de la variable réelle}
Soit $I$ un intervalle de $\mathbb{R}$ et soient $u$ et $v$ des fonctions numériques définies sur $I$.

Si $t$ appartient à $I$, alors $u(t)$ et $v(t)$ sont des nombres réelles et le couple $(u(t),v(t))$ est un élément de $\mathbb{R}^2$
\begin{dfn}
On définit une fonction $f:I\rightarrow \mathbb{R}^2$
\end{dfn}


\section{Les équations différentielles}


On considère deux intervalles ouverts $I$ et $J$ de nombres réels.
$I\subset \mathbb{R}$ et $J\subset \mathbb{R}$.

\begin{dfn}
Le \textit{produit cartésien} de l'intervalle $I$ par l'intervalle $J$ est la partie du plan
$\mathbb{R}^2$
correspondant aux points de coordonnées $(x,y)$ avec $x\in I$ et $y\in J$.
C'est-à-dire l'ensemble $\left\lbrace (x,y) \in \mathbb{R}^2:\; x\in I,\, y\in J \right\rbrace$.
On note $I\times J$.
\end{dfn}
\begin{figure}[hbtp]
\centering
\includegraphics[scale=1]{produit_cartesien.png}
\caption{Le produit cartésien}
\end{figure}

\begin{dfn}
Une fonction $F$ est dite à deux variables si elle est définie sur un produit cartésien de deux intervalles.

Si les deux intervalles sont $I$ et $J$ et $(x,y)\in I \times J$, on note 
$$F:I\times J \rightarrow \mathbb{R}$$
et $F(x,y)$ la valeur de $F$ au point $(x,y)$.
\end{dfn}


\end{document}