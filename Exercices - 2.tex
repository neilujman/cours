\documentclass[10pt,a4paper]{article}
\usepackage[utf8]{inputenc}
\usepackage[french]{babel}
\usepackage[T1]{fontenc}
\usepackage{amsmath}
\usepackage{amsfonts}
\usepackage{amssymb}
\usepackage[left=2cm,right=2cm,top=2cm,bottom=2cm]{geometry}
\author{Julien RITON}
\title{Équations différentielles - Exercices Feuille 2}
\date{12/09/2022}
\begin{document}

\maketitle

\section{La dérivation}

\subsection{Être habile en calcul}
\subsubsection{Exercice sur les fonctions usuelles}
En un point $x$ d'un intervalle $I$ à rappeler, donner la dérivée pour les fonctions $f$ suivantes:
\begin{enumerate}
\item
$f(x)=x^2,\; f(x)=x^3 \; f(x)=x^n\, n\in\mathbb{N}$
\item
$f(x)=\log(x)$ (on note aussi par habitude $ln$ pour logarithme népérien\footnote{Neper était un mathématicien du $XVII^e$ siècle ayant introduit les logarithmes pour faciliter le calcul de multiplication en astronomie, très vite l'application s'est faite en finance. Ainsi avec la table des logarithmes, on transforme une multiplication en une addition}  (ou naturel)
\end{enumerate} 

\subsubsection{Exercice avec de la trigonométrie}
Rappeler la dérivée des fonctions trigonométriques bien connue $\cos$, $\sin$ et $\tan$, en précisant bien l'intervalle sur lequel on peut définir cette dérivée.



\subsubsection{Exercice}

\subsubsection{Exercice}

\subsubsection{Exercice}


\subsubsection{Exercice}


\subsubsection{Exercice}

\subsection{Le calcul inverse : la recherche de primitives}

\subsubsection{Exercice}

\subsubsection{Exercice}

\subsubsection{Exercice}

\subsubsection{Exercice}

\subsubsection{Exercice}

\subsubsection{Exercice}


\end{document}