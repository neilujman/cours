\documentclass[10pt,a4paper]{article}
\usepackage[utf8]{inputenc}
\usepackage[french]{babel}
\usepackage[T1]{fontenc}
\usepackage{amsmath}
\usepackage{amsfonts}
\usepackage{amssymb}
\usepackage{graphicx}
\usepackage[left=2cm,right=2cm,top=2cm,bottom=2cm]{geometry}
\author{Julien RITON}
\title{Sur la séance du 05/09/2022}

\begin{document}
\maketitle

\section{Équations différentielles linéaires à coefficients constants}
\subsection{Équation $y'-ay=0$}
Soit $a$ un nombre réel.
Une solution de l'équation différentielle 
$$y'-ay=0$$
est une fonction $f:\mathbb{R}\rightarrow \mathbb{R}$ dérivable et telle que $f'-af=0$,
c'est-à-dire
$$f'(x) - a f(x) = 0 \qquad \forall x\in \mathbb{R}.$$


La fonction nulle, définie par $f(x)=0$ quel que soit $ x\in \mathbb{R}$, est solution de l'équation différentielle.
Nous avons aussi une autre solution: la fonction $x\mapsto \exp(ax)$\footnote{Nous pouvons aussi noter $\exp(ax)$ par $e^{ax}$.}. En effet, si l'on pose $u(x)=\exp(ax)$, alors il vient $u'(x)=a \exp(ax) = a u(x)$ pour tout $x\in\mathbb{R}$, ou encore $u'-au=0$.

\paragraph{Théorème}
Les solutions de l'équation différentielle $y'-ay=0$ sont les fonctions $x\mapsto \lambda \exp(ax)$, où $\lambda\in\mathbb{R}$.

\subparagraph{Démonstration}
Objet de la séance du 12/09/2022.


\subsection{Équation différentielle $y''+py'+qy=0$}
Soient $p$ et $q$ des nombres réels.
Une solution de l'équation différentielle $y''+py'+qy=0$ est une fonction $f:\mathbb{R}\rightarrow \mathbb{R}$ deux fois dérivable et telle que $f''+pf'+qf=0$, 
c'est-à-dire
$$f''(x) + p f'(x) + q f(x) = 0 \qquad \forall x\in \mathbb{R}.$$

Considérons le polynôme $P\in\mathbb{R}[X]$ défini par $P=X^2+pX+q$.

\paragraph{Théorème}
\begin{itemize}
\item
Si le polynôme $P$ a deux racines réelles distinctes $r$ et $s$, les solutions de l'équation différentielle sont les fonctions $x\mapsto \lambda \exp(rx) + \mu \exp(sx)$ où $\lambda$ et $\mu$ sont des nombres réelles.
\item
Si le polynôme $P$ a une racine réelle doubles $r$, les solutions de l'équation différentielle sont les fonctions $x\mapsto (\lambda x+ \mu )\exp(rx)$ où $\lambda$ et $\mu$ sont des nombres réelles.
\item
Si le polynôme $P$ a deux racines complexes non réelles $r+i\omega$ et $r+i\omega$ où $r$ et $\omega$ sont réels, alors les solutions de l'équation différentielle sont les fonctions $x\mapsto (\lambda \cos(\omega x) + \sin(\omega x) ) \exp(rx)$ où $\lambda$ et $\mu$ sont des nombres réelles. 
\end{itemize}
\subparagraph{Démonstration}
Objet de la séance du 12/09/2022.
\end{document}