\documentclass[10pt,a4paper]{article}
\usepackage[utf8]{inputenc}
\usepackage[french]{babel}
\usepackage[T1]{fontenc}
\usepackage{amsmath}
\usepackage{amsfonts}
\usepackage{amssymb}
\usepackage[left=2cm,right=2cm,top=2cm,bottom=2cm]{geometry}
\author{Julien RITON}
\title{Équations différentielles - Exercices}
\date{05/09/2022}
\begin{document}

\maketitle

\section{Équations différentielles linéaires à coefficients constants}

\subsection{Équations différentielles linéaires d'ordre 1}
\subsubsection{Exercice}
Résoudre l'équation suivante (sur l'intervalle $I=\mathbb{R}$):
$$y'=-y.$$

\subsubsection{Exercice}
Résoudre l'équation suivante (sur l'intervalle $I=\mathbb{R}$):
$$4y'+y=0.$$

\subsubsection{Exercice}
Résoudre l'équation $$y'+3y=0$$.

Donner la solution qui prend la valeur 4 au point $x_0=2$.

\subsubsection{Exercice}
On considère l'équation différentielle 
\begin{equation}\label{eq1}
y'(x)+2y(x)=\exp(3x).
\end{equation}

Résoudre l'équation différentielle intermédiaire obtenue en substituant dans (\ref{eq1}) l'inconnue $y$ en l'inconnue $z$ définie par $y=z+\dfrac{1}{5}\exp(3x)$.

En déduire les solutions de $\left(E\right)$.

\subsubsection{Exercice}
On considère l'équation différentielle $$y'(x)+2y(x)=3*\exp(3x) \left(E\right)$$.
\begin{enumerate}
\item
Déterminer le réel $\alpha$ tel que la fonction $y_0$ définie par $y_0(x)=\alpha\exp(-3x)$
soit solution de $(E)$.
\item
Montrer que $y$ est solution de $(E)$ si et seulement si $y-y_0$ est solution d'une équation différentielle homogène du premier ordre que l'on déterminera.
\item
Déterminer alors toutes les solutions de $(E)$ et préciser celle qui est nulle en $0$.
\end{enumerate}


\subsubsection{Exercice}
Soit $f$ une fonction définie et dérivable sur $\mathbb{R}$ dont la dérivée ne s'annule pas sur $\mathbb{R}$.
Soit $\mathcal{C}$ la courbe représentative de $f$ dans un plan rapporté à un repère orthonormé $(0,\vec{i},\vec{j})$ et soit $M$ un point de $\mathcal{C}$.
On ote $P$ la projection orthogonale de $M$ sur l'axe $(x'x)$ et $Q$ l'intersection de la tangente en $M$ à $\mathcal{C}$ avec l'axe $(x'x)$.
\begin{enumerate}
\item
Montrer que $\overline{PQ}=-1$ pour tout point $M$ de $\mathcal{C}$ si et seulement si $f$ satidfait à une équation différentielle que l'on détermina.
\item
Déterminer les fonctions $f$ vérifiant $\overline{PQ}=-1$ pour tout point $M$ de $\mathcal{C}$.
\end{enumerate}


\subsubsection{Exercice}
Soit $f$ une fonction définie et dérivable sur $\mathbb{R}$ et vérifiant :
\begin{equation}
\label{eq2}
f(x+t)=f(x)*f(t) \; \forall (x,t) \in \mathbb{R}.
\end{equation}

\begin{enumerate}
\item
Montrer que si $f$ vérifie (\ref{eq2}), alors $f$ est solution d'une équation différentielle d'ordre 1 que vous établirez (dériver par rapport à $t$, puis prendre $t=0$).

\item
Déterminer les fonctions $f$ satisfaisant à la relation $\ref{eq2}$.
\end{enumerate}

\subsection{Équations différentielles linéaires d'ordre 2}

\subsubsection{Exercice}
Résoudre l'équation suivante (sur l'intervalle $I=\mathbb{R}$):
$$2y''+y'-y=0.$$

\subsubsection{Exercice}
Résoudre l'équation suivante (sur l'intervalle $I=\mathbb{R}$):
$$16y'-8y'-y=0.$$

\subsubsection{Exercice}
Résoudre l'équation suivante (sur l'intervalle $I=\mathbb{R}$):
$$y''=-5y.$$

\subsubsection{Exercice}
Résoudre l'équation suivante (sur l'intervalle $I=\mathbb{R}$):
$$y''+y'+\dfrac{5}{2}y=0.$$

\subsubsection{Exercice}
\begin{enumerate}
\item
Résoudre l'équation suivante (sur l'intervalle $I=\mathbb{R}$):
$$y''-8y'+15y=0.$$
\item
Déterminer la solution $y_1$ qui vérifie $y_1(0)=1$ et ${y'}_1(0)=2$.
\end{enumerate}

\subsubsection{Exercice}
On considère l'équation différentielle $y''-2y'+y=0$.
\begin{enumerate}
\item
Résoudre cette équation.
\item
Déterminer la solution qui vérifie les deux conditions suivantes:
\begin{itemize}
\item
La courbe représentant cette solution passe par $I(0,4)$.
\item
La tangente à cette courbe au point $I$ a pour coefficient directeur $2$.
\end{itemize}
\end{enumerate}


\end{document}