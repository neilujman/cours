\documentclass[10pt,a4paper]{article}
\usepackage[utf8]{inputenc}
\usepackage[french]{babel}
\usepackage[T1]{fontenc}
\usepackage{amsmath}
\usepackage{amsfonts}
\usepackage{amssymb}
\usepackage[left=2cm,right=2cm,top=2cm,bottom=2cm]{geometry}
\author{Julien RITON}
\title{construction}

\newtheorem{dfn}{Définition}
\newtheorem{thm}{Théorème}
\usepackage{graphicx}
\begin{document}
\section{Équation différentielle - généralités}

On considère deux intervalles ouverts $I$ et $J$ de nombres réels.
$I\subset \mathbb{R}$ et $J\subset \mathbb{R}$.

\begin{dfn}
Le \textit{produit cartésien} de l'intervalle $I$ par l'intervalle $J$ est la partie du plan
$\mathbb{R}^2$
correspondant aux points de coordonnées $(x,y)$ avec $x\in I$ et $y\in J$.
C'est-à-dire l'ensemble $\left\lbrace (x,y) \in \mathbb{R}^2:\; x\in I,\, y\in J \right\rbrace$.
On note $I\times J$.
\end{dfn}
\begin{figure}[hbtp]
\centering
\includegraphics[scale=1]{produit_cartesien.png}
\caption{Le produit cartésien}
\end{figure}

\begin{dfn}
Une fonction $F$ est dite à deux variables si elle est définie sur un produit cartésien de deux intervalles.

Si les deux intervalles sont $I$ et $J$ et $(x,y)\in I \times J$, on note 
$$F:I\times J \rightarrow \mathbb{R}$$
et $F(x,y)$ la valeur de $F$ au point $(x,y)$.
\end{dfn}


\end{document}