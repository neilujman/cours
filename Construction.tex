\documentclass[10pt,a4paper]{article}
\usepackage[utf8]{inputenc}
%\usepackage[french]{babel}
\usepackage[T1]{fontenc}
\usepackage{amsmath}
\usepackage{amsfonts}
\usepackage{amssymb}
\usepackage[left=2cm,right=2cm,top=2cm,bottom=2cm]{geometry}
\author{Julien RITON}
\title{Sur les équations différentielles}

\newtheorem{dfn}{Définition}
\newtheorem{thm}{Théorème}
\usepackage{graphicx}
\begin{document}
\maketitle
\section{Équation différentielle - une première prise en main}

\subsection{Introduction}
\subsubsection{Quelques rappels}
\paragraph{L'usage de la dérivation}
\subparagraph{La dérivation}
Nous rappelons la définition de la dérivée d'une fonction $f$ en un point $x$.
C'est la limite quand $h$ tend vers $0$ du rapport des accroissements entre la valeur de la fonction et la variable et est noté $f'(x)$:
$$
f'(x)=\lim_{h\rightarrow 0} \dfrac{f(x+h)-f(x)}{(x+h)-x}
=\lim_{h\rightarrow 0} \dfrac{f(x+h)-f(x)}{h}
$$
On a aussi les notations suivantes:
$$f'x), \; \dfrac{df(x)}{dx} $$





Souvent, quand on étudie un phénomène qui évolue avec le temps, on choisit comme variable le temps que l'on note par $t$. Les valeurs liées au temps peuvent être les coordonnées d'un point
\subsection{Définition}
\begin{dfn}
Une écriture faisant apparaître une relation entre une fonction, sa variable et au moins l'une de ses dérivées est appelée une équation différentielle.
\end{dfn}
\paragraph{Exemple d'écriture}

\subparagraph{$y'=y$}
L'écriture $y'=y$ est une équation différentielle. Cette écriture fait apparaître une relation entre UNE (et non LA) fonction $y$
et sa dérivée $y'$.

\subparagraph{$y'= 3x+1$}
L'écriture $y'= 3x+1$ est une équation différentielle. On y voit une relation entre la dérivée d'une fonction $y$ et la variable $x$ de $y$. Plus rigoureusement, on devrait écrire $y'(x)=3x+1$.


\subsection{L'équation $y'-ay=0$}




\begin{dfn}
L'écriture suivante, avec $a\in\mathbb{R}$,
$$y'-ay=0$$
est une équation différentielle dite linéaire du premier ordre homogène à coefficient constant.
\end{dfn}

\section{Les équations différentielles}


On considère deux intervalles ouverts $I$ et $J$ de nombres réels.
$I\subset \mathbb{R}$ et $J\subset \mathbb{R}$.

\begin{dfn}
Le \textit{produit cartésien} de l'intervalle $I$ par l'intervalle $J$ est la partie du plan
$\mathbb{R}^2$
correspondant aux points de coordonnées $(x,y)$ avec $x\in I$ et $y\in J$.
C'est-à-dire l'ensemble $\left\lbrace (x,y) \in \mathbb{R}^2:\; x\in I,\, y\in J \right\rbrace$.
On note $I\times J$.
\end{dfn}
\begin{figure}[hbtp]
\centering
\includegraphics[scale=1]{produit_cartesien.png}
\caption{Le produit cartésien}
\end{figure}

\begin{dfn}
Une fonction $F$ est dite à deux variables si elle est définie sur un produit cartésien de deux intervalles.

Si les deux intervalles sont $I$ et $J$ et $(x,y)\in I \times J$, on note 
$$F:I\times J \rightarrow \mathbb{R}$$
et $F(x,y)$ la valeur de $F$ au point $(x,y)$.
\end{dfn}


\end{document}