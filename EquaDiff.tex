\documentclass[11pt]{beamer}
\usetheme{Warsaw}
\usepackage[utf8]{inputenc}
\usepackage[french]{babel}
\usepackage[T1]{fontenc}
\usepackage{amsmath}
\usepackage{amsfonts}
\usepackage{amssymb}
\usepackage{graphicx}
\author{Julien RITON}
\title{Les équations différentielles}
%\setbeamercovered{transparent} 
%\setbeamertemplate{navigation symbols}{} 
%\logo{} 
\institute{UB - I.U.T. de Bourgogne site d'Auxerre} 
\date{05/09/2022} 
\subject{B.U.T. Génie Civil - Construction Durable} 

\newtheorem{dfn}{Définition}
\newtheorem{thm}{Théorème}

\begin{document}

\begin{frame}
\titlepage
\end{frame}

\begin{frame}
\tableofcontents
\end{frame}

\section{Introduction}
\begin{frame}{L'art des mathématiques}
\begin{itemize}
\item
Au lycée : c'est encore du bricolage "simple".
\item
Au niveau des études supérieurs : le bricolage devient plus un art ou du génie.
De plus, il faut chercher à conceptualiser.

\end{itemize}
\end{frame}

\begin{frame}{Niveau de conceptualisation}
\begin{itemize}

\item
Fondamental:
\item
Application:

\end{itemize}

\end{frame}


\section{Les fondations}
\begin{frame}{Les fondations}
\begin{itemize}
\item
L'ensemble des Mathématiques reposent sur fondations bien définies.
\item
Ces fondations sont les axiomes, ce sont des théorèmes de bases mais dont la portée est générale et que nous démontrons pas.
\item
Les théorèmes, propositions, lemmes et corrolaires sont des briques qui reposent sur les axiomes.
\item
Le tout est cimenté par la logique.


\end{itemize}
\end{frame}


\subsection{Les objets}

\subsection{Les axiomes}
Nous n'allons pas énumérées tout les axiomes.

Le plus célèbre est l'axiome du choix.

Même pour travailler avec des nombres entiers, nous avons besoin d'axiomes.




\subsection{La logique}
\subsubsection{L'implication}
\begin{frame}{L'implication}
Une affirmation mathématique $\mathcal{P}$ implique une autre affirmation $\mathcal{Q}$:
dès que $mathcal{P}$ est vraie, $mathcal{Q}$ est aussi vraie.

$$\mathcal{P}\Rightarrow\mathcal{Q}$$
\end{frame}


\subsection{Les propositions}
\begin{frame}{Les théorèmes}
Le plus souvent 
\end{frame}


\section{\'{E}qua diff linéaire d'ordre 1 à coef constants}

\subsection{Une équation}
\begin{frame}{Équation}
Une équation n'est pas à confondre avec une identité.
\begin{dfn}
Une \textbf{identité} est une écriture sous forme d'égalité qui est vraie pour tous les objets présents dans celle-ci.
\end{dfn}
Exemple : l'écriture $a^2-b^2=(a-b)(a+b)$ est une identité
\end{frame}
% =====
\begin{frame}{Équation la plus simple}
\begin{dfn}
Soit $a$ un nombre réel.\\
Une solution de l'équation différentielle 
$$y'-ay=0$$
est une fonction $f:\mathbb{R}\rightarrow\mathbb{R}$ dérivable et telle que $f'-af=0$, c'est-à-dire
$$f'(x)-af(x)=0 \; pour\; tout\; x\; dans \; \mathbb{R}.$$ 
\end{dfn}
\end{frame}
% =====
\begin{frame}{Solutions de $y'-ay=0$}
\begin{thm}
Les solutions de l'équation différentielle $y'-ay=0$ sont les fonctions $f:\mathbb{R}\rightarrow\mathbb{R}$ définies par 
$$f(x)=\lambda \exp(ax)  \; \forall x \in \mathbb{R}$$
pour un $\lambda\in\mathbb{R}$ donné.
\end{thm}
\begin{proof}
On vérifie bien que les $f$ solutionnent l'équation.
Réciproquement chaque solution s'exprime comme la formule.
\end{proof}

\end{frame}

% =====
\begin{frame}{Équation encore assez simple}
\begin{dfn}
Soit $a$ un nombre réel.\\
Une solution de l'équation différentielle 
$$y''+ay=0$$
est une fonction $f:\mathbb{R}\rightarrow\mathbb{R}$ dérivable et telle que $f''+af=0$, c'est-à-dire
$$f''(x)+af(x)=0 \; pour\; tout\; x\; dans \; \mathbb{R}.$$ 
\end{dfn}
\end{frame}
% =====
\begin{frame}{Solutions de $y''+ay=0$}
\begin{thm}
Les solutions de l'équation différentielle $y''+ay=0$ sont les fonctions $f:\mathbb{R}\rightarrow\mathbb{R}$ définies par 
\begin{itemize}
\item si $a=0$ : 
$f(x)=\lambda x +\mu\; \forall x \in \mathbb{R}$, 
\item si $a>0$ : 
$f(x)=\lambda \cos(\alpha x )+\mu \sin(\alpha x )\; \forall x \in \mathbb{R}$,
 avec $\alpha=\sqrt{a}$,
\item si $a<0$ :
$f(x)=\lambda \exp(\alpha x) +\mu \exp(-\alpha x) \; \forall x \in \mathbb{R}$,
 avec $\alpha=\sqrt{-a}$,

\end{itemize}
pour $\lambda\in\mathbb{R}$ et $\mu\in\mathbb{R}$ donnés.

\end{thm}
\begin{proof}
On vérifie bien que les $f$ solutionnent l'équation.
Réciproquement chaque solution s'exprime comme la formule.
\end{proof}

\end{frame}

% =====




\end{document}